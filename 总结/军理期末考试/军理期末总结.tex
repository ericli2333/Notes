\documentclass[lang=cn,newtx,12pt,scheme=chinese]{elegantbook}

\title{2023年春季军事理论期末梳理}
\subtitle{致敬考古专家}

\author{Eric Li}
\institute{NJU ICS}
\date{\today}
\version{Private}
% \bioinfo{自定义}{信息}

\extrainfo{注意:内部资料请勿外传,使用本文带来的一切后果作者概不负责}

\setcounter{tocdepth}{3}

\logo{logo-blue.png}
\cover{cover.png}
% \cover{yx.jpg}

% 本文档命令
\usepackage{array}
\newcommand{\ccr}[1]{\makecell{{\color{#1}\rule{1cm}{1cm}}}}

% 修改标题页的橙色带
\definecolor{customcolor}{RGB}{32,178,170}
% \colorlet{coverlinecolor}{customcolor}
\usepackage{cprotect}

\addbibresource[location=local]{reference.bib} % 参考文献,不要删除

\begin{document}

\maketitle
\frontmatter

\tableofcontents

\mainmatter
\chapter{试论新军事革命的主要内容及战争形态未来发展趋势}
\section{新军事革命的主要内容\footnote{见教材P127,其中新军事革命主要内容为第三部分130页}}
\subsection{新军事革命的内涵}
所谓的新军事革命特指在工业社会走向信息社会的时代,以信息技术为核心并得以广泛应用,从而引起军事领域武器装备、军事理论和组织体制等一系列的根本变革,导致彻底改变战争形态和军队建设模式的一场革命。\label{sec:新军事革命定义}\footnote{教材P128页}

\subsection{新军事革命的五个方面}
\begin{enumerate}
  \item 创新军事技术,升级武器装备的信息化
  \item 创新军事力量体系,重塑军队组织形态
  \item 创新军事理论,推动军队建设转型
  \item 创新人才培养体系,打造高素质军事人才队伍
  \item 创新作战样式,适应新的战争形态
\end{enumerate}\label{enum:新军事革命的五个方面}

\section{战争形态未来发展趋势\footnote{本部分为综合论述部分,书中没有明确答案,可参考第4、5章P115页开始}}
\subsection{ChatGPT 版本}
未来的战争形态将会受到多种因素的影响,包括科学技术的进步、制造业的发展、国际政治形势的变化等。以下是未来战争形态的若干发展趋势:
\begin{enumerate}
  \item 信息化战争。随着信息技术的发展,新型的信息化战争已成为各国军队不可避免的选择。信息化战争会以信息(网络攻击、电磁干扰等)为主要手段,对敌方的信息系统进行破坏和干扰,从而达到战略目的。
  \item 网络战争。随着互联网的广泛应用,网络战争已成为战争发展的重要方向。网络战争不仅涉及到网络空间,也波及到电网、水网、交通网络等各种基础设施。
  \item 非对称战争。随着国际关系的复杂化,非对称战争的形式也在不断地出现。具有技术先进、武器装备精良的大国将可能面对来自恐怖分子、游击队等非正规军事力量的挑战。在这种情况下,重视抗打击能力和反应速度的灵活性将变得更加重要。
  \item 智能化战争。在计算机与智能化技术的发展下,武器装备将更加智能化,军事作战将会更加自动化、信息化。未来的战争将会是由人与机器的协同作战,信息感知、数据处理能力将成为一个重要指标。
\end{enumerate}

\subsection{教材可能有用的话}
\begin{enumerate}
  \item 当今世界,战争形态向信息化深度演进,智能化战争初见端倪,混合战争频繁呈现,作战方式深刻变革,战争空间全面扩展,新概念,新技术迭代更新。\marginpar{P121}
  \item 信息化战争是信息社会或信息时代的主要战争形态。它是在机械化战争进入成熟期,随着信息社会的到来而逐渐成熟的,发展上答题经过由低级到高级的数字化、网络化、智能化三个阶段。\marginpar{P127} 
  \item 随着人工智能的迅猛发展,作为战略性前沿技术,人工智能已经展现出了极大的军事应用潜力和前景。未来中远期内,信息化战争将步入其更高层次的发展阶段——智能化战争时代。人工智能将极大提高武器和军事技术装备的作战效能,颠覆性改变当前的的作战样式和战争形态。可以预测,在未来战争中,智能化将成为战争的致胜因素,智能化武器装备将成为路、海、空、天、网络、电磁等所有战场空间的主战武器装备并起到决胜作用。\marginpar{P128}
  \item 未来的信息化战争的五种主要方式:\label{def:信息战的方式}\marginpar{P140}
    \begin{enumerate}
      \item 精确作战
      \item 太空战
      \item 网络战  
      \item 无人化作战
      \item 信息心理战
    \end{enumerate}
  \item 信息化战争发展趋势:
    \begin{enumerate}
      \item 战争理论创新发展\marginpar{P150}
      \item 武器装备智能升级\marginpar{P151}
      \item 组织形态变化革新\marginpar{P151}
      \item 作战样式异彩纷呈\marginpar{P151}
      \item 作战要素一体融合\marginpar{P152}
    \end{enumerate}
\end{enumerate}

\section{我的回答}
新军事革命是指在工业社会走向信息社会的时代,以信息技术为核心并得以广泛应用,从而引起军事领域武器装备、军事理论和组织体制等一系列的根本变革,导致彻底改变战争形态和军队建设模式的一场革命。\ref{sec:新军事革命定义}
主要包括创新军事技术,升级武器装备的信息化;创新军事力量体系,重塑军队组织形态;创新军事理论,推动军队建设转型;创新人才培养体系,打造高素质军事人才队伍;创新作战样式,适应新的战争形态,这五个方面。\ref{enum:新军事革命的五个方面}

我认为,战争形态的未来发展趋势总体而言是趋于信息化、智能化的。
随着人工智能的迅猛发展,作为战略性前沿技术,人工智能已经展现出了极大的军事应用潜力和前景。未来中远期内,信息化战争将步入其更高层次的发展阶段——智能化战争时代。
人工智能将极大提高武器和军事技术装备的作战效能,颠覆性改变当前的的作战样式和战争形态。
可以预测,在未来战争中,智能化将成为战争的致胜因素,智能化武器装备将成为路、海、空、天、网络、电磁等所有战场空间的主战武器装备并起到决胜作用。

在作战方式上,未来的战争可能有几种作战模式:精确作战、太空战、网络战、无人化作战、信息心理战等方式。\ref{def:信息战的方式}

在作战理念上,战争理论创新发展,提出了混合战争的作战理论,构建全新信息化战争理论体系。

在作战装备上,将发展信息化装备,发展海、陆、空信息化作战平台。\footnote{最后一章}

\chapter{简述习近平强军思想的主要内容及地位作用\footnote{教材P151}}
\section{习近平强军思想的主要内容}
习近平的强军思想主要集中体现为十个明确\footnote{教材P117}:

\begin{enumerate}
  \item 明确强国必须强军,巩固国防和强大人民军队是新时代坚持和发展中国特色社会主义、实现中华民族伟大复兴的战略支撑。
  \item 明确党在新时代的强军目标是建设一支听党指挥、能打胜仗、作风优良的人民军队,必须同国家现代化进程相一致。
  \item 明确党对军队绝对领导是人民军队建军之本、强军之魂,必须全面贯彻党领导军队的一系列根本原则和制度,确保部队绝对忠诚、绝对纯洁、绝对可靠。
  \item 明确军队是要准备打仗的,必须聚焦能打仗,打胜仗,创新发展军事战略指导,构建中国特色现代作战体系。
  \item 明确作风优良是我军鲜明特色和政治优势,必须加强作风建设、纪律建设,坚定不移正风肃纪、反腐惩恶,大力弘扬我党我军光荣传统和优良作风,永葆人民军队性质、宗旨、本色。
  \item 明确推进强军事业必须坚持政治建军、改革强军、科技兴军、依法治军,更加注重聚焦实战、更加注重创新驱动、更加注重体系建设、更加注重集约高效、更加注重军民融合,全面提高革命化现代化正规化水平。
  \item 明确改革是强军的必由之路,必须推进军队组织形态现代化,构建中国特色现代军事力量体系,完善中国特色社会主义军事制度。
  \item 明确创新是引领发展的第一动力,必须坚持向科技创新要战斗力,统筹推进军事理论、技术、组织、管理、文化等各方面创新,建设创新型人民军队。
  \item 明确现代化军队必须构建中国特色军事法治体系,推动治军方式根本性转变,提高国防和军队建设法治化水平。
  \item 明确军民融合发展是兴国之举、强军之策,必须坚持发展和安全兼顾、富国和强军统一,形成全要素、多领域、高效益军民融合深度发展格局,构建一体化的国家战略体系和能力。
\end{enumerate}

\section{重要地位}
开辟了马克思主义军事理论中国化时代化的新境界,擘画了全面建成世界一流军队的宏伟蓝图,引领了新时代人民军队的伟大变革,强固了全军官兵奋斗强军的精神支柱。\marginpar{P120}
\end{document}